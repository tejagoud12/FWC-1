\documentclass[12pt]{article}
\usepackage{graphicx}
%\documentclass[journal,12pt,twocolumn]{IEEEtran}
\usepackage[none]{hyphenat}
\usepackage{graphicx}
\usepackage{enumitem}
\usepackage{listings}
\usepackage[english]{babel}
\usepackage{graphicx}
\usepackage{caption}
\usepackage{hyperref}
\usepackage{booktabs}
\usepackage{array}
\usepackage{amsmath}   % for having text in math mode
\usepackage{listings}
\lstset{
  frame=single,
  breaklines=true
}
%New macro definitions
\newcommand{\mydet}[1]{\ensuremath{\begin{vmatrix}#1\end{vmatrix}}}
\providecommand{\brak}[1]{\ensuremath{\left(#1\right)}}
\providecommand{\norm}[1]{\left\lVert#1\right\rVert}
\newcommand{\solution}{\noindent \textbf{Solution: }}
\newcommand{\myvec}[1]{\ensuremath{\begin{pmatrix}#1\end{pmatrix}}}
\let\vec\mathbf

\begin{document}
\begin{center}
\textbf\large{CLASS-11\\CHAPTER-11 \\ CONIC SECTIONS}
\end{center}

\section*{EXERCISE - 11.1}
ln each of the following exercises  1 to 5, find the equation of the circle with
\begin{enumerate}
\item centre $(0,2)$ and radius 2
\item centre $(-2,3)$ and radius 4
\item centre $ \brak{\frac{1}{2},\frac{1}{4}} \text{ and } radius \frac{1}{12}$
\item centre $(1,1)$ and radius$\sqrt{2}$
\item centre $(-a,-b)$ and radius$\sqrt{a^2-b^2}$
\end{enumerate}
ln each of the following exercises 6 to 9; fing the centre and radius of the circles.
\begin{enumerate}[resume]
\item $(x+5)^2+(y-3)^2=36$
\item $x^2+y^2-4x-8y-45=0$
\item $x^2+y^2-8x+10y-12=0$
\item $2x^2+2y^2-x=0$
\item Find the equation of the circle passing through the points $(4,1)$  and  $(6,5)$ and whose centre is on the line $4x+y=16$.
\item Find the equation of the circle passihg though the points $(2,3)$  and  $(-1,1)$  and  whose centre is an the line $x-3y-11=0$.
\item Find the equation of the circle with radius 5 whose centre lies on $x$-axis and passes through the point $(2,3)$.
\item Find the equation of the circle passing through $(0,0)$ and making intercepts $a$ and $b$ an the coordinate axes.
\item Find the equation of a circle  with centre $(2,2)$ and passes through the point $(4,5)$.
\item Does the point $(-2.5,3.5)$ lie inside,outside or on the circle $x^2+y^2=25?$
\end{enumerate}
\section*{EXERCISE - 11.2}
ln each of the following exercises 1 to 6, find the coordinates of the focus, axis of the parabola, the equation of the directrix and the  length of the latus rectum.
\begin{enumerate}
\item $y^2=12x$
\item $x^2=6y$
\item $y^2=8x$
\item $x^2=-16y$
\item $y^2=10x$
\item $x^2=-9y$
\end{enumerate}
ln each of the exercises 7 to 12, find the equation of the parabol that satisfies the given conditions.
\begin{enumerate}[resume]
\item Focus $(6,0)$;  directix $x$=-6
\item Focus $(0,-3)$; directix $y$=3
\item vertex $(0,0)$; focus $(3,0)$
\item vertex $(0,0)$; focus $(-2,0)$
\item vertex $(0,0)$ passing through $(5,2)$  and  symmetric with respect to $y$-axis.
\end{enumerate}
\section*{EXERCISE - 11.3}
ln each of the exercises 1 to 9, find the coordinates of the foci, the vertices, the coordinates of the foci, the vertices, the length of major axis, the minor axis, the eccentricity and the length of the latus rectum of the ellipse.
\begin{enumerate}
\item $\frac{x^2}{36}+\frac{y^2}{16}=1$
\item $\frac{x^2}{4}+\frac{y^2}{25}=1$
\item $\frac{x^2}{16}+\frac{y^2}{9}=1$
\item $\frac{x^2}{25}+\frac{y^2}{100}=1$
\item $\frac{x^2}{49}+\frac{y^2}{36}=1$
\item $\frac{x^2}{100}+\frac{y^2}{400}=1$
\item $36x^2+4y^2=144$
\item $16x^2+y^2=16$
\item $4x^2+9y^2=36$
\end{enumerate}
ln each of the following exercises 10 to 20, find the equation for the ellipse that satisfies the given conditions;
\begin{enumerate}[resume]
\item vertices $(\pm5,0)$, foci$(\pm4,0)$
\item vertices $(\pm13) $  foci$(\pm5)$
\item vertices $(\pm6,0)$ foci$(\pm4,0)$
\item Ends of major axis $(\pm3,0),$ ends of minor axis $(\pm2)$
\item Ends of major axis $(0,\pm\sqrt{5})$ , ends of minor axis $(\pm1,0)$
\item Length of major axis 26, foci $(\pm5,0)$
\item Length  of minor axis 16, foci $(\pm,6)$
\item Foci $(\pm3,0)$, $a$=4
\item $b$=3, $c$=4, centre at the origin; foci on the $x$ axis.
\item centre at $(0,0),$ major axis on the $y$-axis and passes through the points $(3,2)$ and $(1,6).$
\item major axis on the $x$-axis and passes through the points $(4,3)$  and  $(6,2).$
\end{enumerate}
\section*{EXERCISE - 11.4}
ln each of the exercises 1 to 6, find the coordinates of the foci and the vertices, the eccentricity and the length of the latus rectum  of the hyperbolas.
\begin{enumerate}
\item $\frac{x^2}{16}$-$\frac{y^2}{9}$=1
\item $\frac{y^2}{9}$-$\frac{x^2}{27}$=1
\item $9y^2-4x^2=36$
\item $16x^2-9y^2=576$
\item $5y^2-ax^2=36$
\item $49y^2-16x^2=784.$
\end{enumerate}
ln each of the exercises 7 to 15, find the equations of the  hyperbola satisfying the given conditions.
\begin{enumerate}[resume]
\item vertices $(\pm2,0),$ foci $(\pm3,0)$
\item vertices $(0,\pm5),$ foci $(0,\pm8)$
\item vertices $(0,\pm3),$ foci $(0,\pm5)$
\item Foci $(\pm5,0),$ the transverse axis is of length 8.
\item Foci $(0,\pm13),$ the conjugate axis is of length 24.
\item Foci $(\pm3\sqrt{5},0),$ the latus rectum is of length 8.
\item Foci $(\pm4,0),$ the latus rectum is of length 12
\item vertices $(\pm7,0),$ $e$=$\frac{4}{3}.$
\item Foci $(0,\pm\sqrt{10})$, passing through $(2,3)$
\end{enumerate}
\section*{Miscellaneous Exercise on chapter 11}
\begin{enumerate}		
\item lf a parabolic reflector is 20 cm in diameter and 5 cm deep . find the focus.
\item An arch is in the form of a parabola with its axis vertical. The arch is 10 m high and 5 m wide at the base. How wide is it  2 m from the vertex of the parabola ?
\item The cable of a uniformy loaded suspension bridge hangs in the form of a parabola. The roadway which is  horizontal and 100 m long is supported by vertical wires attached to the cable, the longest wire being 30 m and the shortest beihg 6 m. Find the lenth of a supporting wire attached to the  roadway 18 m from the middle.
\item An arch is in the form of a semi-ellipse. lt is 8 m wide and 2 m high at the centre Find the height of the arch at a point 1.5m from one end.
\item A rod of length 12cm maves with its ends always touchihg the coordinate axes. Determine the equation of the lous of a point $\vec{P}$ on the rod, which is 3 cm from the end in contact with the $x$-axis
\item Find the area of the triangle formed by the lines joining the vertex of the parabola $x^2=12y$ to the ends of its latus rectum.
\item A man running a racecourse nates that sum of the distances from the two flag posts from him is always 10 m  and the distance between the flag posts is 8 m. Find the equatian of the posts traced by the man.
\item An equilateral triangle is inscribed in the paraboia $y^2=4ax$, where one vertex is at the vertex of the paraboia. Find the length of the side of the triangle.
\end{enumerate}
\end{document}
