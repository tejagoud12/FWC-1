\documentclass[12pt]{article}
\usepackage{graphicx}
%\documentclass[journal,12pt,twocolumn]{IEEEtran}
\usepackage[none]{hyphenat}
\usepackage{graphicx}
\usepackage{enumitem}
\usepackage{listings}
\usepackage[english]{babel}
\usepackage{graphicx}
\usepackage{caption}
\usepackage{hyperref}
\usepackage{booktabs}
\usepackage{array}
\usepackage{amsmath}   % for having text in math mode
\usepackage{listings}
\lstset{
  frame=single,
  breaklines=true
}
%New macro definitions
\newcommand{\mydet}[1]{\ensuremath{\begin{vmatrix}#1\end{vmatrix}}}
\providecommand{\brak}[1]{\ensuremath{\left(#1\right)}}
\providecommand{\norm}[1]{\left\lVert#1\right\rVert}
\newcommand{\solution}{\noindent \textbf{Solution: }}
\newcommand{\myvec}[1]{\ensuremath{\begin{pmatrix}#1\end{pmatrix}}}
\let\vec\mathbf

\begin{document}
\begin{center}
\textbf\large{CLASS-11\\CHAPTER-10 \\ CONIC SECTIONS}
\end{center}

\section*{EXERCISE - 10.3}
In each of the exercises 1 to 9, find the coordinates of the foci, the vertices, the length of major axis, the minor axis, the eccentricity and the length of the latus rectum of the elipse
\begin{enumerate} 
\item $\frac{x^2}{36}+\frac{y^2}{16}$=1
\item $\frac{x^4}{4}+\frac{y^2}{25}$=1
\item $\frac{x^2}{16}+\frac{y^2}{9}$=1
\item $\frac{x^2}{25}+\frac{y^2}{100}$=1
\item $\frac{x^2}{49}+\frac{y^2}{36}$=1
\item $\frac{x^2}{100}+\frac{y^2}{400}$=1
\item $36x^2+4y^2=144$
\item $16x^2+y^2=16$
\item $4x^2+9y^2=36$
\end{enumerate}
In each of the following  Exercises 10 to 20, find the equatian for the ellipse that satisfies the given conditions:
\begin{enumerate}[resume]
\item vertices $(\pm5,0)$, foci$(\pm4,0)$
\item vertices$(\pm13)$, foci$(\pm5)$
\item vertices$(\pm6,0)$, foci$(\pm4,0)$
\item Ends of major axis $(\pm3,0)$, ends of minor axis $(0,\pm2)$
\item Ends of major axis $(0,\pm\sqrt{5})$, ends of minor axis$(\pm1,0)$
\item Length of major axis 26, foci$(\pm5,0)$
\item Length of minor axis 16, foci $(0,\pm6)$
\item Foci $(\pm3,0), a=4$
\item $b=3, c=4,$ centre at the origin; foci on the $x$-axis.
\item  Centre at $(0,0)$ major axis on the $y$-axis and passes through the points $(3,2)$ and $(1,6)$
\item Major axis on the $x$ -axis and passes through the points $(4,3)$ and $(6,2)$.
\end{enumerate}
\end{document}
